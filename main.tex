% - - - - - - - - - - - - - - - - - - - - - - - - - - - - - - - - - - - - - - %
% @brief   Example of using Share LaTeX
% @author  Sigve Karolius (SK)
% @date    2017-04-07  (started)
% @copyright 2017 SK
% @details
% - - - - - - - - - - - - - - - - - - - - - - - - - - - - - - - - - - - - - - %
% API for accessing custom meta-data
\providecommand{\myauthor}{S.~Karolius}
\providecommand{\myffiliation}{NTNU}
\providecommand{\mycopyright}{SK}
\providecommand{\mycomment}{%
This is a short abstract outlining the scope and findings in this work.%
}
\providecommand{\mydate}{1.~januar~2010}
\providecommand{\myfirstpage}{123}
\providecommand{\myinstitution}{Dept. of Chem. Eng.,~Norwegian Univ. of Sci. and Tech.}
\providecommand{\myjournal}{ABCD of Modelling}
\providecommand{\mykeywords}{list, of, keywords}
%\providecommand{\mylistings}{my-listings}%      My personal listings package
\providecommand{\mymonth}{April}
\providecommand{\mynumber}{1}
%\providecommand{\mynotation}{my-glossary}% My personal glossary and notation
\providecommand{\mytitle}{Modelling as a Core Activity}
\providecommand{\myvolume}{XVI}
\providecommand{\myyear}{2017}
% --------------------------------------------------------------------------- %
% Preamble...
% --------------------------------------------------------------------------- %
\documentclass{my-memoir}
%
%\usepackage{lettrine}
%asdfsadf
% --------------------------------------------------------------------------- %
% Start document...
% --------------------------------------------------------------------------- %
%\newsavebox\ltmcbox
\begin{document}
%
\maketitle
%
%\printglossaries
%
\raggedcolumns
\begin{multicols}{2}
%
\tableofcontents
%
\mergeglossaries%         <--  Merge Greek and Lating Symbols into one glossary
\printglossary[type=main, style=mylong2col, title=Symbols]
\printglossary[type=\acronymtype]
% --------------------------------------------------------------------------- %
\section{Introduction}
%
Linear models\index{M!Models} can be represented as follows:
%
\begin{align}
  \dot{\State} &= \matState \State + \matInput \Input 
\end{align}
%
%\begin{table}[H]
% \footnotesize
% \caption{%
%   Table%
% }
% \centering
% \begin{tabular}{l l} \toprule
%    a & B \\ \midrule \midrule
%    c & d \\ \bottomrule
% \end{tabular}
%\end{table}
%
% --------------------------------------------------------------------------- %
\section{Thermodynamics}
%
In thermodynamics \Gibbs is an eponym which takes its name from from \glsentrylong{Gibbs}. \Gibbs-energy is defined as:
%
\[
  \Gibbs := U - TS - \NegativePressure V
\]
% --------------------------------------------------------------------------- %
\section{Numerical Methods}
%
\begin{lstlisting}[%
 language=python,%
 caption={[Simple Newton Raphson]%
 The "Newton-step" is performed in Line \ref{line:butanol}%
 },%
 label={lst:everest},%
 escapeinside={<@}{@>},%
]
def nr(f,dfdx,x0,tol=1e-12,maxit=13):
  """ 1D Newton-Raphson solver """
  xk=x0; iflag=False; cflag=False; i=1
  while not iflag:
    <@\textcolor{red}{xkp1 = xk - f(xk)/dfdx(xk)}\label{line:butanol}@> # NR step
    cflag = abs(xkp1 - xk) <= tol
    if not cflag:
        xk = xkp1; i += 1
        iflag = i > maxit
        continue
    return xk
  return xk
\end{lstlisting}
%
\end{multicols}
%
\clearpage
%
\twocolindex
\printindex
%
\end{document}
